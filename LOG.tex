\documentclass{report}
\title{Daily Project Log}
\author{Garrett Smith}

\usepackage{hyperref}

\begin{document}

\section*{2012-05-15}
\begin{itemize}
  \item
    Completed installing and updating development machine;
  \item
    Drew initial flow diagram for Android application;
  \item
    Installed and began using Latex;
  \item 
    Setup Eclipse, Eclim, and Vim to work together;
  \item
    Installed Android emulator and setup Android virtual device (AVD);
  \item
    Wrote two example applications to get the feel for Android development;
\end{itemize}

\section*{2012-05-16}
\begin{itemize}
  \item
    Created a git repository on external server for version control;
  \item
    Setup debugging to run on real Android device;
  \item
    Created a simple gallery following a tutorial at \href{http://developer.android.com/resources/tutorials/views/hello-gallery.html}{Android Developers};
  \item
    Created a simple drawing demo without the use of tutorials to learn and use the api;
\end{itemize}
\subsection*{Challenges}
\begin{itemize}
  \item
    Vim was not providing enough support for programming the application so I have moved to using Eclipse with Vim embedded in it;
\end{itemize}

\section*{2012-05-17}
\begin{itemize}
  \item
    Created an about screen;
  \item
    Began loading images from the gallery application, will extend this to request pictures from camera as well which is a relatively simple addition;
  \item
    Cleaned up code and refactored to be more in line with what should be used in the project;
\end{itemize}

\section*{2012-05-18}
\begin{itemize}
  \item
    Setup selection to scale and center to screen size;
  \item
    Learned how to use the, sanity-saving, debugger;
  \item
    Added moving the selection area using touch, will need to be improved quite abit;
\end{itemize}
\subsection*{Challenges}
\begin{itemize}
  \item
    Working with matrix transformations has been a day-long adventure/headache but I'm getting the hang of them;
\end{itemize}

\section*{2012-05-22}
\begin{itemize}
  \item 
    Refactored code to be more readable, scale images properly, seperate neghbourhoods and images into seperate classes and handle rotated images;
  \item
    Rotation is now read from the picture attributes to handle portrait photos;
  \item
    Neighbourhoods now draw after from their own class;
  \item
    Can now move neighbourhood properly by tracking touch events;
  \item
    Moving the neighbourhood is constrained to the image;
\end{itemize}
\subsection*{Challenges}
\begin{itemize}
  \item 
    Working between scales can be quite a challenge, but is neccessary to scale and pan the image properly;
\end{itemize}

\section*{2012-05-23}
\begin{itemize}
  \item 
    Fixed points touched on the screen not being translated to points relative to the image;
  \item
    Added simple neighbourhood resizing, you can only resize one edge at a time and I feel like it could be written in a much more elegant manner;
  \item
    Added edge pair resizing but I feel there must be a cleaner way to do it all;
  \item
    Constrained resize to image boundries;
  \item
    Added basic pinch zoom support, doesn't maintain aspect ratio properly when you sacle a side to the boundry of the image and then scale back;
  \item
    You can now move while you scale; 
  \item
    There is a minimum size you can shrink to;
  \item
    Prevented flipping the neighbourhood over itself while resizing;
  \item
    Scale no longer is now indepndant of axis, this felt much better;
  \item
    Position of neighbourhood is no longer lost on rotate;
\end{itemize}

\section*{2012-05-24}
\begin{itemize}
  \item 
    Added shape selection menu, not yet functional;
  \item
    Added preliminary circle neighbourhoods, more oval right now;
  \item
    Fixed reading exif info;
  \item
    Neighbourhood is restored properly again;
  \item
    Change neighbourhood drawing to only require a dirty rectangle redraw rather than redraw the entire view, this speeds up redraws the for smaller neighbourhoods;
  \item
    Got polygon neihgbourhood drawing working;
  \item
    Added initial bitmap scalling, still needs to be tested;
\end{itemize}
\subsection*{Challenges}
\begin{itemize}
  \item 
    It would appear UI events run in their own thread. So, when I would restore the selected shape selection, on resuming or rotating the screen, it would reset my restored bounds for the shape. I solved this be creating a runnable, an object encapsulating some code, that is created when bounds are found to restore and ran and destroyed after a shape is selected if the runnable object exists;
\end{itemize}

\section*{2012-05-25}
\begin{itemize}
  \item 
    Drew handles on the neighbourhoods;
  \item
    Centered focussed neighbourhood after moving and resizing, this needs some tweaking so it will only occur when there is a significant change;
\end{itemize}
\subsection*{Challenges}
\begin{itemize}
  \item 
    Attempted to add zooming methods by scaling a user matrix and then concaterating the changes but had a difficult time zooming and panning at the same time, needs further work;
\end{itemize}

\section*{2012-05-28}
\begin{itemize}
  \item 
    Wrote developing guide so others can get started quickly;
  \item
    Finally got view to follow the current neighbourhood smoothly while you move and resize it;
\end{itemize}

\section*{2012-05-29}
\begin{itemize}
  \item 
    Cleaned up large amounts of code and refactored;
  \item
    Added Creating and moving the points of polygon neighbourhoods;
  \item
    Added removing points from the polygon by draggin them outside of the image;
  \item
    Created a polgon class for use with neighbourhoods;
\end{itemize}

\section*{2012-05-30}
\begin{itemize}
  \item 
    Fleshed out polygon implementation;
  \item
    Improved neighbourhood drawing and added highlighting points that will be removed;
  \item
    Implemented algorithm to check if point is within a polygon;
  \item
    Added moving polygon neighbourhoods;
  \item
    Added resizing polygons;
  \item
    Points added to polygons are now added between the nearest point and the nearest neighbour of that point, this makes adding points to a polygon much more intuative but doesn't seem to quite solve the problem;
\end{itemize}

\end{document}
