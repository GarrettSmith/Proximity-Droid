\documentclass{article}
\title{Installing the Android Development Environment}
\author{Garrett Smith}

\usepackage{hyperref}
\usepackage{listings}

\begin{document}

\section{Installing Software}

\subsection{Eclipse}
\begin{enumerate}
  \item
    \href{http://www.eclipse.org/downloads/}{Download Eclipse};
  \item
    Install and setup the default options it asks;
\end{enumerate}

\subsection{ADT}
\begin{enumerate}
  \item
    Download and install ADT following these \href{http://developer.android.com/sdk/eclipse-adt.html#downloading}{directions};
  \item
    When asked after installing ADT, choose to download and install the Android SDK;
\end{enumerate}

\subsection{Android SDK}
You can follow the \href{http://developer.android.com/sdk/installing.html}{full guide} at Android Developers.
\begin{enumerate}
  \item
    Start the Android SDK Manager from Eclipse by goind to \textbf{Window > Android SDK Manager} or by clicking the button in the toolbar;
  \item
    You will need to install both packages in the tools subdirectory, as well as the most recent Android platform, as of writting 4.0.3;
  \item
    You may also want to install the lowest supported version, as of writting 2.1;
  \item 
    If you are on windows and want to use usb debugging you will need to follow \href{http://developer.android.com/sdk/win-usb.html}{Google USB Driver};
\end{enumerate}

\subsection{AVD}
\begin{enumerate}
  \item
    Start the ADV Manager from either the toolbar or from \textbf{Window > AVD Manager};
  \item
    To create a new AVD press \textbf{New...}, name the AVD, and select the target you want the device to run;
  \item
    You can start your AVD from the AVD Manager by selecting the device and pressing \textbf{Start...} then \textbf{Launch};
\end{enumerate}

\section{Cloning the Git Repository}
This will not tell you how to work with git just how to get the current source code to work with.
\begin{enumerate}
  \item
    You will need some form of git running on your computer. If you are on Windows you may want to try \href{http://help.github.com/win-set-up-git/}{Download and install the latest version of Git for Windows}.
  \item
    Move to the directory you wisth the source code to be downloaded to;
  \item
    To clone the repository run the following command;
    \begin{lstlisting}
      git clone http://git.pixelpeg.ca/compare.git
    \end{lstlisting}
  \item
    Within Eclipse, create a new project by going to \textbf{File > New > Android Project};
  \item
    Select \textbf{Create project from existing source} and select the location where you downloaded the source;
  \item
    Press finish and you're done;
\end{enumerate}

\section{Coding Style}
Coding style primarily follows \href{http://source.android.com/source/code-style.html}{Code Style Guidelines for Contributors} with the exception that 2 spaces are used to indent code.

\end{document}
